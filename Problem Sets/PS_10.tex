\documentclass[11pt]{article}
\title{Problem Set 10}
%% Language and font encodings
\usepackage[english]{babel}
\usepackage[utf8x]{inputenc}
\usepackage[T1]{fontenc}

\usepackage{helvet}

%% Sets page size and margins
\usepackage[letterpaper,top=3cm,bottom=2cm,left=3cm,right=3cm,marginparwidth=1.75cm]{geometry}

%% Useful packages
\usepackage{amsmath}
\usepackage{graphicx}
\usepackage{tcolorbox}
\usepackage{amssymb}
\usepackage{amsthm}
\usepackage{lastpage}
\usepackage{accents}
\usepackage{multicol}

% For better list numbering
\usepackage[shortlabels]{enumitem}

% Font
% \usepackage{tgbonum}


% Tikz
\usepackage{tikz}

\usetikzlibrary{calc,fit,shapes.misc,backgrounds}
\usepackage{pgfplots}
\pgfplotsset{compat = newest}
\usetikzlibrary{positioning, arrows.meta}
\usepgfplotslibrary{fillbetween}

% Headers
\usepackage{fancyhdr}
\pagestyle{fancy}

% Store \@title as \thetitle
\makeatletter
\let\thetitle\@title
\makeatother

\fancyhf{}
\lhead{\fontfamily{qbk}\fontsize{10}{11}\selectfont ECON 3070}
\rhead{\fontfamily{qbk}\fontsize{10}{11}\selectfont \thetitle}
\rfoot{\fontfamily{qbk}\fontsize{10}{11}\selectfont \thepage}


% Sections and Subsections

% define colors
\definecolor{buff-gold}{HTML}{CFB87C}
\definecolor{buff-grey}{HTML}{565A5C}
% custom tcolorbox
\tcbset{colframe=buff-gold, colback=white!100!black}

% new page per section
\usepackage{titlesec}
\newcommand{\sectionbreak}{\clearpage}
% change style of section
\usepackage{sectsty}
\sectionfont{\color{buff-gold} \fontfamily{qbk}\selectfont}
\subsectionfont{\color{buff-grey} \fontfamily{qbk}\selectfont}


\begin{document}
  
\section*{Chapter 10}

\begin{enumerate}
  \item Suppose that in the market for cigarettes, market supply is given by $Q_S = P - 2$, and market demand is given by $Q_D = 30 - P$ (it's a very small market).

  \begin{enumerate}
    \item Find the market equilibrium price and quantity in this market.

    \item Find the consumer surplus and producer surplus in this market.

    \item What is the total surplus in the market?
  \end{enumerate}

  Now suppose that the local government has decided to impose an excise tax of $\$4$ on every unit of the good sold.

  \begin{enumerate}
    \item[(d)] What will be the equilibrium quantity after the tax is imposed? What price will consumers pay? What price will producers receive?

    \item[(e)] What is the incidence of this tax? In other words, how much does the consumer's price increase, and how much does the producer's price decrease?

    \item[(f)] What does the tax incidence suggest about the elasticities of supply and demand?

    \item[(g)] Find the consumer surplus and producer surplus in this market, after the tax is imposed.

    \item[(h)] How much tax revenue does this excise tax generate?

    \item[(i)] What is the total deadweight loss that results from the tax?

    \item[(j)] What is the source of the deadweight loss that results from the tax?
  \end{enumerate}


  \item Suppose that in the market for vaccines, market supply is given by $P_{S}=9+2 Q_{S}$ and market demand is given by $P_{D}=45-Q$. 
  \begin{enumerate}
    \item Find the market equilibrium price and quantity in this market.

    \item Find the consumer surplus and producer surplus in this market.

    \item What is the total surplus in the market?
  \end{enumerate}

  Now suppose that the local government has decided to impose a subsidy of $\$ 6$ on every unit of the good sold, in order to encourage locals to get vaccinated for their own benefit.
  
  \begin{enumerate}
    \item[(d)] What will be the equilibrium quantity after the subsidy is imposed? What price will consumers pay? What price will producers receive?

    \item[(e)] What is the incidence of this subsidy? In other words, how much does the consumer's price decrease, and how much does the producer's price increase?

    \item[(f)] Find the consumer surplus and producer surplus in this market, after the subsidy is imposed.

    \item[(g)] How much is money is this subsidy costing the government?

    \item[(h)] What is the total deadweight loss that results from the subsidy?

    \item[(i)] What is the source of the deadweight loss that results from the subsidy?
  \end{enumerate}

  \item In a perfectly competitive market, the market demand curve is given by $Q_D = 200 - 5P$, and the market supply curve is given by $Q_S = 15P$.

  \begin{enumerate}
    \item Find the equilibrium market price and quantity demanded and supplied in the absence of price controls.
    
    \item Draw supply and demand curves for this market, with quantity on the $x$-axis and price on the $y$-axis. Be sure to label the supply and demand curves. Additionally, mark the equilibrium price and quantity, and indicate which area of the graph represents consumer surplus, and which portion represents producer surplus.

    \item Find the consumer and producer surplus in this market (\emph{Note:} You may want to invert the supply and demand curves first, so that you have $P$ as a function of $Q$).
  \end{enumerate}

  Now suppose that a price ceiling of \$5 per unit is imposed on the market. 

  \begin{enumerate}
    \item[(d)] What is the quantity supplied with a price ceiling of this magnitude? What is the size of the shortage created by the price ceiling?

    \item[(e)] On a separate graph from part (b), redraw the supply and demand curves. Indicate the original equilibrium price and quantity, as well as the new equilibrium price and quantity under the price ceiling. Finally, indicate the areas of the graph that represents producer and consumer surplus, assuming that the individuals with the highest willingness to pay are the ones that buy the good after the ceiling is imposed.

    \item[(f)] Find the consumer and producer surplus in the market after the price ceiling was imposed, under the assumption from part (e).
  \end{enumerate}
\end{enumerate}
\end{document}
