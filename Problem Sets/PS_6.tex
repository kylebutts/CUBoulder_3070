\documentclass[11pt]{article}
\title{Problem Set 6}
%% Language and font encodings
\usepackage[english]{babel}
\usepackage[utf8x]{inputenc}
\usepackage[T1]{fontenc}

\usepackage{helvet}

%% Sets page size and margins
\usepackage[letterpaper,top=3cm,bottom=2cm,left=3cm,right=3cm,marginparwidth=1.75cm]{geometry}

%% Useful packages
\usepackage{amsmath}
\usepackage{graphicx}
\usepackage{tcolorbox}
\usepackage{amssymb}
\usepackage{amsthm}
\usepackage{lastpage}
\usepackage{accents}
\usepackage{multicol}

% For better list numbering
\usepackage[shortlabels]{enumitem}

% Font
% \usepackage{tgbonum}


% Tikz
\usepackage{tikz}

\usetikzlibrary{calc,fit,shapes.misc,backgrounds}
\usepackage{pgfplots}
\pgfplotsset{compat = newest}
\usetikzlibrary{positioning, arrows.meta}
\usepgfplotslibrary{fillbetween}

% Headers
\usepackage{fancyhdr}
\pagestyle{fancy}

% Store \@title as \thetitle
\makeatletter
\let\thetitle\@title
\makeatother

\fancyhf{}
\lhead{\fontfamily{qbk}\fontsize{10}{11}\selectfont ECON 3070}
\rhead{\fontfamily{qbk}\fontsize{10}{11}\selectfont \thetitle}
\rfoot{\fontfamily{qbk}\fontsize{10}{11}\selectfont \thepage}


% Sections and Subsections

% define colors
\definecolor{buff-gold}{HTML}{CFB87C}
\definecolor{buff-grey}{HTML}{565A5C}
% custom tcolorbox
\tcbset{colframe=buff-gold, colback=white!100!black}

% new page per section
\usepackage{titlesec}
\newcommand{\sectionbreak}{\clearpage}
% change style of section
\usepackage{sectsty}
\sectionfont{\color{buff-gold} \fontfamily{qbk}\selectfont}
\subsectionfont{\color{buff-grey} \fontfamily{qbk}\selectfont}


\begin{document}
  
\section*{Chapter 6}

\begin{enumerate}
  \item For each of the production functions below, answer the following questions:
  \begin{enumerate}[(i)]
    \item What is the marginal product of each of the inputs?
    
    \item Does the marginal product of $L$ diminish, remain constant, or increase as the level of $L$ increases?
    
    \item What is the marginal rate of substitution ($MRTS$) of $L$ and $K$?
    
    \item Is the $MRTS_{L,K}$ diminishing, constant, or increasing as the firm substitutes more $L$ for $K$, holding the level of output constant?
    
    \item Does the production function exhibit increasing, constant, or decreasing returns to scale? Show your work.
  \end{enumerate}

  \begin{enumerate}[(a)]
    \item $Q(K, L) = 6 L^{\frac{1}{2}} K^{\frac{1}{2}}$
      
    \vspace*{70mm}
    \item $Q(K, L) = 24L - 1/2 L^2 + 30K - K^2$
    
    \vspace*{70mm}
    \item $Q(K, L) = \sqrt{L} + K$
    
    \vspace*{70mm}
    \item $Q(K, L) = K + 2L$
  \end{enumerate}

  \newpage
  \item Suppose that a firm originally has the production function $Q(K, L) = 10L + 10K$. Over time as the company learns, the production function changes to $Q(K, L) = 40L + 20K$.
  
  \begin{enumerate}[(a)]
    \item Show that the innovation has resulted in technological progress in the sense defined in the notes.
    
    \vspace*{40mm}
    \item Is the technological progress neutral, labor-saving, or capital-saving? How can you tell?
  \end{enumerate}

  \vspace*{40mm}
  \item Suppose that a firm originally has the production function $Q(K, L) = \sqrt{KL}$. Over time as the company learns, the production function changes to $Q(K, L) = \sqrt{K} L$.
  
  \begin{enumerate}[(a)]
    \item Show that the innovation has resulted in technological progress in the sense defined in the notes.
    
    \vspace*{40mm}
    \item Is the technological progress neutral, labor-saving, or capital-saving? How can you tell?
  \end{enumerate}

  \newpage
  \item Suppose that a firm originally has the production function $Q(K, L) = KL$. Over time as the company learns, the production function changes to $Q(K, L) = K^2 L^2$.
  
  \begin{enumerate}[(a)]
    \item Show that the innovation has resulted in technological progress in the sense defined in the notes.
    
    \vspace*{40mm}
    \item Is the technological progress neutral, labor-saving, or capital-saving? How can you tell?
  \end{enumerate}

  \newpage
  \item Let $T$ represent car tires and $F$ represent car frames. The production of a car requires 4 tires and 1 frame. 
  \begin{enumerate}[(a)]
    \item Draw at least two isoquants for car production, with tires on the horizontal axis and frames on the vertical axis. Be sure to label the axes and the isoquants.
    
    \vspace*{100mm}
    \item Write the production function in mathematical notation, $Q(T, F)$, for the car producer's production function.
  \end{enumerate}
\end{enumerate}

\end{document}
