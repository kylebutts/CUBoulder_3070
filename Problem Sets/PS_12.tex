\documentclass[11pt]{article}
\title{Problem Set 12}
%% Language and font encodings
\usepackage[english]{babel}
\usepackage[utf8x]{inputenc}
\usepackage[T1]{fontenc}

\usepackage{helvet}

%% Sets page size and margins
\usepackage[letterpaper,top=3cm,bottom=2cm,left=3cm,right=3cm,marginparwidth=1.75cm]{geometry}

%% Useful packages
\usepackage{amsmath}
\usepackage{graphicx}
\usepackage{tcolorbox}
\usepackage{amssymb}
\usepackage{amsthm}
\usepackage{lastpage}
\usepackage{accents}
\usepackage{multicol}

% For better list numbering
\usepackage[shortlabels]{enumitem}

% Font
% \usepackage{tgbonum}


% Tikz
\usepackage{tikz}

\usetikzlibrary{calc,fit,shapes.misc,backgrounds}
\usepackage{pgfplots}
\pgfplotsset{compat = newest}
\usetikzlibrary{positioning, arrows.meta}
\usepgfplotslibrary{fillbetween}

% Headers
\usepackage{fancyhdr}
\pagestyle{fancy}

% Store \@title as \thetitle
\makeatletter
\let\thetitle\@title
\makeatother

\fancyhf{}
\lhead{\fontfamily{qbk}\fontsize{10}{11}\selectfont ECON 3070}
\rhead{\fontfamily{qbk}\fontsize{10}{11}\selectfont \thetitle}
\rfoot{\fontfamily{qbk}\fontsize{10}{11}\selectfont \thepage}


% Sections and Subsections

% define colors
\definecolor{buff-gold}{HTML}{CFB87C}
\definecolor{buff-grey}{HTML}{565A5C}
% custom tcolorbox
\tcbset{colframe=buff-gold, colback=white!100!black}

% new page per section
\usepackage{titlesec}
\newcommand{\sectionbreak}{\clearpage}
% change style of section
\usepackage{sectsty}
\sectionfont{\color{buff-gold} \fontfamily{qbk}\selectfont}
\subsectionfont{\color{buff-grey} \fontfamily{qbk}\selectfont}


\begin{document}
  
\section*{Chapter 12}

\begin{enumerate}

  \item Suppose that a monopolist faces the demand curve $P(Q)=24-Q$, and has total cost curve $T C(Q)=Q^{2}$

  \begin{enumerate}
    \item[(a)] If the firm is unable to price discriminate, find the firm's profit maximizing price and quantity.

    \item[(b)] On a graph with quantity on the x-axis and price on the y-axis, plot the firm's marginal cost curve, marginal revenue curve, and the demand curve. Indicate the firm's profit maximizing price and quantity. Additionally, identify the areas on the graph that correspond to the consumer surplus, producer surplus, and deadweight loss.

    \item[(c)] Short answer: What is the source of deadweight loss in this market?

    \item[(d)] Find the consumer surplus and producer surplus in the market.

    \item[(e)] Now suppose that the firm gains information on all of it's customers, such that it is able to engage in first-degree price discrimination. How many units will the firm sell?

    \item[(f)] On a separate graph from part (b), with quantity on the $x$-axis and price on the $y$-axis, plot the firm's new marginal revenue curve, along with the marginal cost curve and the demand curve. Indicate the firm's profit maximizing quantity on the graph. Additionally, identify the areas on the graph that correspond to the consumer surplus and producer surplus.

    \item[(g)] Find the consumer and producer surplus in the market, under first-degree price discrimination.    
     
    \item[(h)] What is the deadweight loss in the market now? Explain why it changed in the way that it did.
  \end{enumerate}

  \item Suppose that a monopolist faces the demand curve $P(Q)=40-Q$, and has total cost curve $T C(Q)=F+\frac{3}{2} Q^{2}$, where $F$ is a fixed cost.

  \begin{enumerate}
    \item[(a)] If the firm is unable to price discriminate, find the firm's profit maximizing price and quantity.

    \item[(b)] Find the consumer surplus and producer surplus in the market.

    \item[(c)] For what values of $F$ can a profit-maximizing firm charging a uniform price earn at least zero economic profit?

    \item[(d)] Now suppose that information exists which would allow the firm to perfectly price discriminate (that is, engage in first-degree price discrimination). If the firm had access to this information, and could engage in first-degree price discrimination, what quantity would the firm sell?

    \item[(e)] What would be the producer surplus in the market?

    \item[(f)] What is the maximum amount that this firm would be willing to pay for this information? Why?

    \item[(g)] If the firm were able to perfectly price discriminate, for what values of $F$ could the firm earn at least zero economic profit?
  \end{enumerate}

  % \item Suppose that a firm sells to only one consumer, and that the consumer's demand is given by $P=140 - Q$. The firm's marginal costs are given by $MC = 20$.

  % \begin{enumerate}
  %   \item[(a)] If the firm charges a single price for all units of the good, what is the firm's profitmaximizing price and quantity?

  %   \item[(b)] Find the consumer and producer surplus.

  %   \item[(c)] Now suppose that the firm has realized that they can potentially earn more profit from this consumer by using a block pricing scheme with two blocks. If the firm were to use a block pricing scheme, what would be the profit-maximizing price and quantity for each block?

  %   \item[(d)] On a graph with quantity on the x-axis and price on the y-axis, plot the firm's marginal cost curve, and the demand curve. Identify the profit-maximizing prices and quantities for both blocks. Also, identify the area(s) on the graph that represent consumer and producer surplus.

  %   \item[(e)] Find the consumer and producer surplus in the market, assuming that the firm uses the block-pricing scheme.
  % \end{enumerate}

  \item Consider a market with 100 identical individuals, each with the demand schedule for electricity of $P = 10 - Q$. They are served by an electric utility that operates with a constant marginal cost of 2.
  
  \begin{enumerate}
    \item[(a)] If the firm charged a single price (no price discrimination), what would be the profit maximizing market price and quantity sold? (Note: Since the 100 consumers are identical, and marginal cost is constant, you do not need to aggregate their demand curves to find the optimal price and quantity. You can simply optimize for a single consumer, and then aggregate across consumers.)

    \item[(b)] Find the consumer and producer surplus in the market under this single price. (Note: remember that there are 100 consumers)

    \item[(c)] Now suppose that the firm is considering imposing a two-part tariff, with an upfront subscription fee and a per-unit price. In that case, at what per-unit price should they sell a unit of electricity? How many units should they sell to each customer?

    \item[(d)] What should the price of the subscription fee be in order for the firm to maximize profit?

    \item[(e)] Find the consumer and producer surplus in the market under this new two-part tariff.
  \end{enumerate}


  \item Suppose that Acme Pharmaceutical Company discovers a drug that cures the common cold. Acme has plants in both the United States and Europe and can manufacture the drug on either continent at a marginal cost of 10. Assume there are no fixed costs. In Europe, the demand for the drug is $Q_{E}=70-P_{E}$, where $Q_{E}$ is the quantity demanded when the price in Europe is $P_{E}$. In the United States, the demand for the drug is $Q_{U}=110-P_{U}$, where $Q_{U}$ is the quantity demanded when the price in the United States is $P_{U}$.
  \begin{enumerate}
    \item[(a)] If the firm can engage in third-degree price discrimination, what price should it set on each continent to maximize its profit?

    \item[(b)] Find the producer and consumer surplus in each market.

    \item[(c)] Assume now that it is illegal for the firm to price discriminate, so that it can charge only a single price $\mathrm{P}$ on both continents. Find the aggregate demand curve for the two markets.

    \item[(d)] Assuming that the price will be such that both firms buy some amount of the good under a single price, what price will the firm charge, and what quantity will it sell in the aggregate market?

    \item[(e)] Compare the total consumer surplus and producer surplus with and without price discrimination. Do total consumer and producer surplus increase or decrease when the firm is no longer able to price discriminate?
  \end{enumerate}
\end{enumerate}
\end{document}
