\documentclass[11pt]{article}
\title{Problem Set 4}
%% Language and font encodings
\usepackage[english]{babel}
\usepackage[utf8x]{inputenc}
\usepackage[T1]{fontenc}

\usepackage{helvet}

%% Sets page size and margins
\usepackage[letterpaper,top=3cm,bottom=2cm,left=3cm,right=3cm,marginparwidth=1.75cm]{geometry}

%% Useful packages
\usepackage{amsmath}
\usepackage{graphicx}
\usepackage{tcolorbox}
\usepackage{amssymb}
\usepackage{amsthm}
\usepackage{lastpage}
\usepackage{accents}
\usepackage{multicol}

% For better list numbering
\usepackage[shortlabels]{enumitem}

% Font
% \usepackage{tgbonum}


% Tikz
\usepackage{tikz}

\usetikzlibrary{calc,fit,shapes.misc,backgrounds}
\usepackage{pgfplots}
\pgfplotsset{compat = newest}
\usetikzlibrary{positioning, arrows.meta}
\usepgfplotslibrary{fillbetween}

% Headers
\usepackage{fancyhdr}
\pagestyle{fancy}

% Store \@title as \thetitle
\makeatletter
\let\thetitle\@title
\makeatother

\fancyhf{}
\lhead{\fontfamily{qbk}\fontsize{10}{11}\selectfont ECON 3070}
\rhead{\fontfamily{qbk}\fontsize{10}{11}\selectfont \thetitle}
\rfoot{\fontfamily{qbk}\fontsize{10}{11}\selectfont \thepage}


% Sections and Subsections

% define colors
\definecolor{buff-gold}{HTML}{CFB87C}
\definecolor{buff-grey}{HTML}{565A5C}
% custom tcolorbox
\tcbset{colframe=buff-gold, colback=white!100!black}

% new page per section
\usepackage{titlesec}
\newcommand{\sectionbreak}{\clearpage}
% change style of section
\usepackage{sectsty}
\sectionfont{\color{buff-gold} \fontfamily{qbk}\selectfont}
\subsectionfont{\color{buff-grey} \fontfamily{qbk}\selectfont}


\begin{document}
  
\section*{Chapter 4}

\begin{tcolorbox}[title= {\bf Optimal Consumption:}]
  Consumers will consume until the tangency condition is satisfied:

  $$\frac{\frac{\partial}{\partial x} U(x,y)}{\frac{\partial}{\partial y} U(x,y)} =  \frac{MU_x}{p_x} = \frac{MU_y}{p_y} \hspace{5mm} \text{(Tangency Condition)} $$

  \vspace{2.5mm}
  If the two are not equal, you can shift consumption and increase happiness. For a simple example, assume $p_x = p_y = 1$ and at a given consumption bundle:
  
  $$MU_x = 5 > 3 = MU_y$$ 
  
  \vspace{2.5mm}
  Giving up 1 unit of good $y$ will lose 3 utility but it allows you to buy 1 unit of good $x$ and gain 5 utility, making you happier without spending extra money.
\end{tcolorbox}

\begin{enumerate}
  \item Suppose that Sam can buy only two goods with her income, bread ($B$) and eggs ($E$). Sam wants to buy the combination of bread and eggs that maximizes her utility, which is given by the following function:
  $$ 
    U(B,E) = 20B - \frac{1}{2} B^2 + 40E - \frac{1}{2} E^2
  $$

  \begin{enumerate}[(a)]
    \item Find Sam's marginal utility functions for both bread and eggs. Is Sam's marginal utility for eggs increasing, decreasing, or constant?
    

    \vspace*{50mm}
    \item Given her marginal utility functions, write the optimality condition that Sam's consumption of bread and eggs must satisfy. Explain why this condition must be satisfied in order for Sam's utility to be maximized
  
  \end{enumerate}
\end{enumerate}    
  
\newpage
    \begin{tcolorbox}[title= {\bf Budget Constraints}]
      When consumers are purchasing, they face the constraint of their wallet. They can't spend more than they have in money. Budget constraints are of the form: 
      
      $$p_x x + p_y y \leq I$$ 
      
      which means the amount I spend on $x$ plus the amount I spend on $y$ is less than or equal to budget. Note the most you can spend is when the equation is equal.
      \vspace{-2mm}
    \end{tcolorbox}

\begin{enumerate}
  \item[] 
  \begin{enumerate}[(a)] 
    \item[(c)] Let the price of bread be $P_B$, the price of eggs be $P_E$, and Sam's income be $I$.Write Sam's budget constraint

    \vspace*{20mm}
    \item[(d)] Using Sam's budget constraint, and her optimality condition, solve for Sam's demand curves for bread and eggs, in terms of $P_B$, $P_E$, and $I$. You should finnd that your answer matches the demand curves below. (Note: In this question you will be graded on the steps you took to finnd the answer, so show all steps.)
    
    \newpage
    \item[(e)] Suppose that the price of bread, $P_B$, is $\$1$; the price of eggs, $P_E$, is $\$2$; and Sam's income is $\$40$. How much bread and eggs should Sam consume in order to maximize her utility?
    
    
    \vspace*{50mm}
    \item[(f)] Now, suppose that the price of bread, $P_B$ has changed to $\$2$. The price of eggs, $P_E$, is still $\$2$ and Sam's income is still $\$40$. Given the new price of bread, how much bread and eggs should Sam consume in order to maximize her utility?
    
    \vspace*{50mm}
    \item[(g)] Explain why Sam's consumption of bread and eggs changed in the way that it did
  \end{enumerate}
\end{enumerate}

\newpage
\begin{enumerate}
  \item[3.] Solve the following problems for demand

  \begin{enumerate}[(a)]
    \item $U(x,y) = \min(x, 2y) \text{ and } P_x = 2, P_y = 2, I = 24$
    
    \vspace*{60mm}
    \item $U(x,y) = 2x + 4y \text{ and } P_x = 1, P_y = 3, I = 18$
    
    \vspace*{60mm}
    \item $U(x,y) = x^{\frac{1}{2}} y^{\frac{1}{2}}  \text{ and } P_x = 4, P_y = 4, I = 24$
  \end{enumerate}
\end{enumerate}

\end{document}
