\documentclass[11pt]{article}
\title{Problem Set 7}
%% Language and font encodings
\usepackage[english]{babel}
\usepackage[utf8x]{inputenc}
\usepackage[T1]{fontenc}

\usepackage{helvet}

%% Sets page size and margins
\usepackage[letterpaper,top=3cm,bottom=2cm,left=3cm,right=3cm,marginparwidth=1.75cm]{geometry}

%% Useful packages
\usepackage{amsmath}
\usepackage{graphicx}
\usepackage{tcolorbox}
\usepackage{amssymb}
\usepackage{amsthm}
\usepackage{lastpage}
\usepackage{accents}
\usepackage{multicol}

% For better list numbering
\usepackage[shortlabels]{enumitem}

% Font
% \usepackage{tgbonum}


% Tikz
\usepackage{tikz}

\usetikzlibrary{calc,fit,shapes.misc,backgrounds}
\usepackage{pgfplots}
\pgfplotsset{compat = newest}
\usetikzlibrary{positioning, arrows.meta}
\usepgfplotslibrary{fillbetween}

% Headers
\usepackage{fancyhdr}
\pagestyle{fancy}

% Store \@title as \thetitle
\makeatletter
\let\thetitle\@title
\makeatother

\fancyhf{}
\lhead{\fontfamily{qbk}\fontsize{10}{11}\selectfont ECON 3070}
\rhead{\fontfamily{qbk}\fontsize{10}{11}\selectfont \thetitle}
\rfoot{\fontfamily{qbk}\fontsize{10}{11}\selectfont \thepage}


% Sections and Subsections

% define colors
\definecolor{buff-gold}{HTML}{CFB87C}
\definecolor{buff-grey}{HTML}{565A5C}
% custom tcolorbox
\tcbset{colframe=buff-gold, colback=white!100!black}

% new page per section
\usepackage{titlesec}
\newcommand{\sectionbreak}{\clearpage}
% change style of section
\usepackage{sectsty}
\sectionfont{\color{buff-gold} \fontfamily{qbk}\selectfont}
\subsectionfont{\color{buff-grey} \fontfamily{qbk}\selectfont}


\begin{document}
  
\section*{Chapter 7}

\begin{enumerate}
  \item Suppose the production function of airframes is characterized by a CES production function $Q(K, L) = (L^{\frac{1}{2}} K^{\frac{1}{2}})^2$. The marginal products are given by $MP_L = (L^{\frac{1}{2}} K^{\frac{1}{2}}) L^{-\frac{1}{2}}$ and $MP_K = (L^{\frac{1}{2}} K^{\frac{1}{2}}) K^{-\frac{1}{2}}$.

  \begin{enumerate}[(a)]
    \item Write the optimality conditional that this firm's choice of labor and capital must satisfy in order to be cost-minimizing. 
    
    \vspace*{40mm}
    \item Now suppose that the price of labor is $\$10$/unit and the price of capital is $\$1$/unit. Find the cost minimizing combination fo labor and capital for an airframe manufacturer that wants to produce $121,000$ aiframes (Hint: Note that the optimality condition can be simplified first.)
  \end{enumerate}


  \newpage
  \item Suppose that the production of airframes is instead characterized by a Cobb-Douglas produc- tion function where $Q(K, L) = KL$.
  
  \begin{enumerate}[(a)]
    \item Find the marginal products of capital and labor for this production function.
    

    \vspace*{40mm}
    \item Write the optimality condition that this firm’s choice of labor and capital must satisfy in order to be cost-minimizing.
    
    \vspace*{40mm}
    \item Using the optimality condition and the production function, find the firm’s conditional input demand functions for labor and capital.
    

    \newpage
    \item Now suppose the the price of labor is $\$10$/unit and the price of capital is $\$1$/unit. Find the cost minimizing combination of labor and capital for an airframe manufacturer that wants to produce $121,000$ airframes.
  \end{enumerate}

  
  \newpage
  \item The processing of payroll for the $10,000$ workers in a large firm can either be done using $1$ hour of computer time (denoted by $K$) and no clerks or with $10$ hours of clerical time (denoted by $L$) and no computer time. Computers and clerks are perfect substitutes; for example, the firm could also process its payroll using 1/2 hour of computer time and 5 hours of clerical time.
  
  \begin{enumerate}
    \item[(a)] With $L$ on the $x$-axis and $K$ on the $y$-axis, sketch the isoquant that shows all combinations of clerical time and computer time that allows the firm to process the payroll for $10,000$ workers.
  \end{enumerate}

  Now suppose computer time costs $\$5$ per hour and clerical time costs $\$7.50$ per hour.

  \begin{enumerate}
    \item[(b)] On the graph from part (a), draw at least three isocost lines for the firm for various levels of total cost. Also, indicate the slope of the isocost lines.

    \newpage
    \item[(c)] What are the cost-minimizing choices of $L$ and K?

    \vspace*{40mm}
    \item[(d)] What is the minimized total cost of processing the payroll?

    \vspace*{40mm}
    \item[(e)] Explain, in terms of the optimality condition, why the firm chooses to use only one input, rather than some of each input.

    \vspace*{40mm}
    \item[(f)] Suppose that the price of clerical time remains at $\$7.50$ per hour. How high would the price of an hour of computer time have to be before the firm would find it worthwhile to use only clerks to process the payroll?
  \end{enumerate}

  \newpage
  \item Suppose that a firm produces an output good with the production function $Q = KL$, where $Q$ is the number of units of output per hour when the firm uses $K$ machines and hires $L$ workers each hour.
  
  \begin{enumerate}
    \item[(a)] Find the firm’s marginal product functions for labor and capital, and write the firm’s optimality condition.
  \end{enumerate}


  \vspace*{60mm}
  Now suppose that the firm is currently using $K = 16$ units of capital, and just enough labor to produce $Q = 32$ units of output. The factor price of $K$ is 4 and the factor price of $L$ is 2.

  \begin{enumerate}
    \item[(b)] How much labor is the firm using to produce it’s output? Show that, at this input combination, the optimality condition does not hold.

    \newpage
    \item[(c)] Using the firm’s optimality condition and production function, find the cost minimizing level of capital and labor.

    \vspace*{100mm}
    \item[(d)] By how much does the firm’s total cost ($TC$) fall when the adjust $K$ and $L$ to their cost-minimizing levels?
  \end{enumerate}


\end{enumerate}

\end{document}
