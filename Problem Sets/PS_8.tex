\documentclass[11pt]{article}
\title{Problem Set 8}
%% Language and font encodings
\usepackage[english]{babel}
\usepackage[utf8x]{inputenc}
\usepackage[T1]{fontenc}

\usepackage{helvet}

%% Sets page size and margins
\usepackage[letterpaper,top=3cm,bottom=2cm,left=3cm,right=3cm,marginparwidth=1.75cm]{geometry}

%% Useful packages
\usepackage{amsmath}
\usepackage{graphicx}
\usepackage{tcolorbox}
\usepackage{amssymb}
\usepackage{amsthm}
\usepackage{lastpage}
\usepackage{accents}
\usepackage{multicol}

% For better list numbering
\usepackage[shortlabels]{enumitem}

% Font
% \usepackage{tgbonum}


% Tikz
\usepackage{tikz}

\usetikzlibrary{calc,fit,shapes.misc,backgrounds}
\usepackage{pgfplots}
\pgfplotsset{compat = newest}
\usetikzlibrary{positioning, arrows.meta}
\usepgfplotslibrary{fillbetween}

% Headers
\usepackage{fancyhdr}
\pagestyle{fancy}

% Store \@title as \thetitle
\makeatletter
\let\thetitle\@title
\makeatother

\fancyhf{}
\lhead{\fontfamily{qbk}\fontsize{10}{11}\selectfont ECON 3070}
\rhead{\fontfamily{qbk}\fontsize{10}{11}\selectfont \thetitle}
\rfoot{\fontfamily{qbk}\fontsize{10}{11}\selectfont \thepage}


% Sections and Subsections

% define colors
\definecolor{buff-gold}{HTML}{CFB87C}
\definecolor{buff-grey}{HTML}{565A5C}
% custom tcolorbox
\tcbset{colframe=buff-gold, colback=white!100!black}

% new page per section
\usepackage{titlesec}
\newcommand{\sectionbreak}{\clearpage}
% change style of section
\usepackage{sectsty}
\sectionfont{\color{buff-gold} \fontfamily{qbk}\selectfont}
\subsectionfont{\color{buff-grey} \fontfamily{qbk}\selectfont}


\begin{document}
  
\section*{Chapter 8}

\begin{enumerate}
  \item Suppose that a firm has the production function $Q(K,L) = KL$, with $MP_L = K$ and $MP_K = L$. Furthermore, suppose that, in the short run, the firm’s level of capital is fixed at $\bar{K} = 5$.
  
  \begin{enumerate}[(a)]
    \item Find the firm’s short-run, cost-minimizing quantity of labor $L^{*}_{SR}$ as a function of $Q$ and $w$.
    
    \item Now suppose that the price of labor and the price of capital are both $1$. That is, $w = 1$ and $r = 1$. Find the firm’s short-run total cost function $TC_{SR}$. Identify which portion of the firm’s total cost function represents the firm’s fixed costs, and which portion represents the firm’s variable costs.
    
    \item On a graph with $Q$ on the horizontal axis, and dollars on the vertical axis, plot the firm’s fixed cost curve ($SFC$), variable cost curve ($SVC$), and total cost curve ($STC$). Be sure to label the curves, as well as the axes.

    \item Find the optimality condition that the firm’s choice of capital and labor must solve in order for it to be cost-minimizing in the long run. Use this optimality condition, along with the firm’s production function, to find the firm’s conditional labor and capital demand equations as functions of $\bar{Q}, r,$ and $w$

    \item Now, use the prices given above, write the firm’s long-run total cost function $TC_{LR}$. Identify which portion of the firm’s total cost function represents the firm’s fixed costs, and which portion represents the firm’s variable costs.

    \item On a separate graph from part (c), plot the firm’s short-run total cost curve and long-run total cost curve.

    \item Now suppose that the price of capital has increased from $r = 1$ to $r = 4$. Find the firm’s new short-run total cost function and long-run total cost function.

    \item On the same graph as part (f), plot the firm’s new short-run and long-run total cost curves.
  \end{enumerate}
  
  \newpage
  \item Suppose that a firm has the production function $Q(K, L) = \sqrt{KL}$. In the short run suppose that $K = 4$. Also, suppose that the price of labor is given by $w = 8$, and the price of capital is given by $r = 2$
  
  \begin{enumerate}
  
    \item Find the firm’s short-run, cost-minimizing level of labor $L^∗_{SR}$.
  
  \item Find the firm’s short-run total cost function $TC_{SR}$. Identify which part of the total cost function pertains to the firm’s fixed cost ($SFC$), and which part pertains to the firm’s variable costs ($SVC$).
  
  \item On a single graph, plot the firm’s short-run fixed cost curve, variable cost curve, and total cost curve ($STC$). Label each of these curves, as well as the x- and y- axes.
  
  \item Using the firm’s short-run total cost function, find the firm’s short-run average total cost function ($SAC$), and short-run marginal cost function ($SMC$). Identify which portion of the firm’s average total cost function pertains to the firm’s short-run average variable costs and short-run average fixed costs.
  
  \item Given the prices above, write the optimality condition that the firm’s long-run choice of $K$ and $L$ must satisfy. Use this optimality condition, along with the firm’s production function, to find the firm’s conditional labor and capital demand functions ($L^∗_{LR}$ and $K^∗_{LR}$).
  
  \item Find the long-run firm’s total cost function. Identify which part of the total cost function pertains to the firm’s fixed cost ($FC$), and which part pertains to the firm’s variable costs ($VC$)
  
  \item Find the firm’s long-run average cost and long-run marginal cost functions.
  
  \item Does the firm’s production function exhibit increasing, decreasing, or constant economies of scale (assuming that the prices of capital and labor are fixed)?
  
  \item Find the quantity of output at which the firm attains their minimum efficient scale (think of the relationship between the long-run average cost curve and the long-run marginal cost curve).
  \end{enumerate}
\end{enumerate}

\end{document}
