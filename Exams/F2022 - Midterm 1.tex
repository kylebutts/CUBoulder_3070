\documentclass[11pt]{article}
\title{Midterm 1}
%% Language and font encodings
\usepackage[english]{babel}
\usepackage[utf8x]{inputenc}
\usepackage[T1]{fontenc}

\usepackage{helvet}

%% Sets page size and margins
\usepackage[letterpaper,top=3cm,bottom=2cm,left=3cm,right=3cm,marginparwidth=1.75cm]{geometry}

%% Useful packages
\usepackage{amsmath}
\usepackage{graphicx}
\usepackage{tcolorbox}
\usepackage{amssymb}
\usepackage{amsthm}
\usepackage{lastpage}
\usepackage{accents}
\usepackage{multicol}

% For better list numbering
\usepackage[shortlabels]{enumitem}

% Font
% \usepackage{tgbonum}


% Tikz
\usepackage{tikz}

\usetikzlibrary{calc,fit,shapes.misc,backgrounds}
\usepackage{pgfplots}
\pgfplotsset{compat = newest}
\usetikzlibrary{positioning, arrows.meta}
\usepgfplotslibrary{fillbetween}

% Headers
\usepackage{fancyhdr}
\pagestyle{fancy}

% Store \@title as \thetitle
\makeatletter
\let\thetitle\@title
\makeatother

\fancyhf{}
\lhead{\fontfamily{qbk}\fontsize{10}{11}\selectfont ECON 3070}
\rhead{\fontfamily{qbk}\fontsize{10}{11}\selectfont \thetitle}
\rfoot{\fontfamily{qbk}\fontsize{10}{11}\selectfont \thepage}


% Sections and Subsections

% define colors
\definecolor{buff-gold}{HTML}{CFB87C}
\definecolor{buff-grey}{HTML}{565A5C}
% custom tcolorbox
\tcbset{colframe=buff-gold, colback=white!100!black}

% new page per section
\usepackage{titlesec}
\newcommand{\sectionbreak}{\clearpage}
% change style of section
\usepackage{sectsty}
\sectionfont{\color{buff-gold} \fontfamily{qbk}\selectfont}
\subsectionfont{\color{buff-grey} \fontfamily{qbk}\selectfont}


\newtoggle{INCLUDEANSWERS}
\toggletrue{INCLUDEANSWERS}
% \togglefalse{INCLUDEANSWERS}

\newcommand{\answer}[1]{\iftoggle{INCLUDEANSWERS}{{\color{violet!70!white}\textbf{Solution:} #1}}{}}


\begin{document}
\emph{Good luck to you!}
  
\begin{enumerate}
  \item If disposable incomes rise by 5\% and demand changes from 100 units to 105, what is the income elasticity of demand? Interpret this number.
  
  \answer{\emph{(10 pts.)} 
  
    $\% \Delta \text{ in } Q = \frac{105 - 100}{100} = 0.05$

    $$
      \varepsilon_{I, Q} = \frac{\% \Delta \text{ in } Q}{\% \Delta \text{ in } I} = \frac{0.05}{0.05} = 1
    $$

    To intrepret, a 1\% increase in income yields a 1\% increase in quantity demanded.
  }


  \bigskip
  \item The law of demand tells us that when the price of a good goes up, the quantity demanded for that good goes down. Describe why a business might want to know the price elasticity of demand.
  
  \answer{\emph{(10 pts.)}

    The price elasticity of demand tells you \emph{how much} demand decreases with a price increase. This can be used to help optimally price products. If a good is very inelastic, then you can raise prices without losing many sales, for example.
  }
  
  \bigskip
  \item Consider the demand function $x^*(P_x, P_y, I) = \frac{I}{P_x P_y}$
  
  \begin{enumerate}
    \item Does this good satisfy the law of demand?
    \item Is this good a normal good or a inferior good?
    \item Are goods $x$ and $y$ substitutes, complements, or neither?
    \item If the maker of good $y$ raises the price of their good, how does that affect the sales of good $x$?
  \end{enumerate}

  \answer{\emph{(20 pts.)}
    \begin{enumerate}
      \item \emph{(5 pts.)} $\frac{\partial x^*}{\partial P_x} < 0$ (denominator grows, so $x^*$ goes down). Therefore, this good satisfies the law of demand ($P_x \uparrow$ implies $x^* \downarrow$)

      \item \emph{(5 pts.)} $\frac{\partial x^*}{\partial I} = \frac{1}{P_x P_y} > 0$. Therefore $I \uparrow$ implies $x^* \uparrow$. Hence, good $x$ is a normal good.
      
      \item \emph{(5 pts)} ] $\frac{\partial x^*}{\partial P_y} < 0$. Therefore $P_y \uparrow$ implies $x^* \downarrow$ and hence goods $x$ and $y$s are complements. 
      
      \item \emph{(5 pts.)} Since $x$ and $y$ are complements, the demand for good $x$ goes down after an increase in price of $y$.
    \end{enumerate}
  }
  
  \bigskip
  \item For the utility function $U(x,y) = 2x + 4y$, Calculate the marginal rate of substitution $MRS_{x,y}$. Interpret this number in words.
  
  \bigskip
  \item For the following utility functions, draw three indifference curves
  \begin{enumerate}
    \item $U(x,y) = 2x + 4y$
    
    \item $U(x,y) = min(x, 2y)$
  \end{enumerate}

  \answer{\emph{(15 pts.)}
    \begin{enumerate}
      \item \emph{(7.5 pts.)} $\bar{U} = 2x + 4y \implies y = \bar{U}/2 - x/2$. Therefore, indifference curves are a bunch of downward-sloping lines
      
      \item \emph{(7.5 pts.)} "brackets" along the line $x = 2y$.  
    \end{enumerate}
  }

  \bigskip
  \item There are two suppliers in the market, Starbucks and Dunkin, who sell coffee. Their supply curves are given by $Q_{Starbucks}^S = -10 + 2P$ and $Q_{Dunkin}^S = -20 + 2P$. Demand for coffee in Boulder is given by $Q_{mkt}^D = 42 - 2P$.
  
  \begin{enumerate}
    \item Solve for the market supply, $Q_{mkt}^S$. 
    
    \answer{
      $$
        Q_{mkt}^S =
        \begin{cases}
          0        & \text{if } 0 < P < 5 \\
          -10 + 2P & \text{if } 5 \leq P \leq 10 \\
          -30 + 4P & \text{if } P \geq 10
        \end{cases}
      $$
    }
    
    \item What is the equilibrium price and quantity for coffee?
    
    \answer{
      $$
        P^* = 12 \text{ and } Q^* = -30 + 4 * 12 = 18
      $$
    }
    
    \item Suppose the demand curve, shifts out to $Q_{mkt}^D = 60 - 2P$, will the market price go up or go down? (hint: no need for math)
    

  \end{enumerate}

  \answer{\emph{(15 pts.)}
    \begin{enumerate}
      \item \emph{(5 pts.)} 

      $$
        Q_{mkt}^S =
        \begin{cases}
          0        & \text{if } 0 < P < 5 \\
          -10 + 2P & \text{if } 5 \leq P \leq 10 \\
          -30 + 4P & \text{if } P \geq 10
        \end{cases}
      $$

      \item \emph{(5 pts.)} 
      
      Try $-10 + 2P = 42 - 2P \implies P^* = 13$. This would violate $5 \leq P < 10$. 

      Try $-30 + 4P = 42 - 2P \implies P^* = 12$. Plugging into demand gives $Q^* = 42 - 2 * 12 = 18$.

      \item \emph{(5 pts.)} The demand curve goes up, so the equilibrium price increases.
    \end{enumerate}
  }

  \bigskip
  \item Consider the following consumer optimal consumption problem 
  $$
    \max_{x,y} x^{1/5} y^{4/5} \text{ subject to } P_x x + P_y y = I
  $$

  \begin{enumerate}
    \item What is the optimality condition? Interpret in words, why we know that if the consumer is consuming optimally the optimality condition must hold. 
    
    \item Solve for optimal demand $x^*(P_x, P_y, I)$ and $y^*(P_x, P_y, I)$
    
    \item At price $P_x = 5$, $P_y = 10$, and $I = 100$, what is optimal demand for $x$ and $y$?
    
    \item Using the optimal demand, $x^*$, is good $x$ a normal good or an inferior good? How do you know?
    
    \item What is the demand curve for good $x$, $x^*(P_x)$, when $P_y = 10$ and $I = 100$? Draw this curve. From your graph, does this good sastisfy the law of demand?
  \end{enumerate}

  \answer{\emph{(20 pts.)}
    \begin{enumerate}
      \item \emph{(10 pts.)}
      
      $$\frac{MU_x}{P_x} = \frac{MU_y}{P_y} \implies y = 4 \frac{P_x}{P_y} x$$

      \item \emph{(3 pts.)} Pluging the optimality condition into the budget constraint yields
      
      $$
        I = P_x * x + P_y * 4 \frac{P_x}{P_y} x = 5P_x x \implies x^* = \frac{I}{5P_x}
      $$
      
      Similarly, $y^* = \frac{4I}{5P_y}$.
      

      \item \emph{(3 pts.)}
      
      $$
        x^* = \frac{100}{5 * 5} = 4 \text{ and } y^* = \frac{4 * 100}{5 * 10} = 8
      $$


      \item \emph{(2 pts.)}
      
      $$
        \frac{\partial x^*}{\partial I} = \frac{1}{5P_x} > 0 \implies x \text{ is a normal good.}
      $$
      
      \item \emph{(2 pts.)}
      $$ 
        x^*(P_x) = \frac{100}{5P_x} = \frac{20}{P_x}
      $$

      This is a downward sloping curve, which satisfies the law of demand ($P_x \uparrow$ implies $x^* \downarrow$)
      

    \end{enumerate}
  }


\end{enumerate}


\end{document}
