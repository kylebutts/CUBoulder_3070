\documentclass[11pt]{article}
\title{Midterm 2 Prep}
%% Language and font encodings
\usepackage[english]{babel}
\usepackage[utf8x]{inputenc}
\usepackage[T1]{fontenc}

\usepackage{helvet}

%% Sets page size and margins
\usepackage[letterpaper,top=3cm,bottom=2cm,left=3cm,right=3cm,marginparwidth=1.75cm]{geometry}

%% Useful packages
\usepackage{amsmath}
\usepackage{graphicx}
\usepackage{tcolorbox}
\usepackage{amssymb}
\usepackage{amsthm}
\usepackage{lastpage}
\usepackage{accents}
\usepackage{multicol}

% For better list numbering
\usepackage[shortlabels]{enumitem}

% Font
% \usepackage{tgbonum}


% Tikz
\usepackage{tikz}

\usetikzlibrary{calc,fit,shapes.misc,backgrounds}
\usepackage{pgfplots}
\pgfplotsset{compat = newest}
\usetikzlibrary{positioning, arrows.meta}
\usepgfplotslibrary{fillbetween}

% Headers
\usepackage{fancyhdr}
\pagestyle{fancy}

% Store \@title as \thetitle
\makeatletter
\let\thetitle\@title
\makeatother

\fancyhf{}
\lhead{\fontfamily{qbk}\fontsize{10}{11}\selectfont ECON 3070}
\rhead{\fontfamily{qbk}\fontsize{10}{11}\selectfont \thetitle}
\rfoot{\fontfamily{qbk}\fontsize{10}{11}\selectfont \thepage}


% Sections and Subsections

% define colors
\definecolor{buff-gold}{HTML}{CFB87C}
\definecolor{buff-grey}{HTML}{565A5C}
% custom tcolorbox
\tcbset{colframe=buff-gold, colback=white!100!black}

% new page per section
\usepackage{titlesec}
\newcommand{\sectionbreak}{\clearpage}
% change style of section
\usepackage{sectsty}
\sectionfont{\color{buff-gold} \fontfamily{qbk}\selectfont}
\subsectionfont{\color{buff-grey} \fontfamily{qbk}\selectfont}


\newtoggle{INCLUDEANSWERS}
\toggletrue{INCLUDEANSWERS}
% \togglefalse{INCLUDEANSWERS}

\newcommand{\answer}[1]{\iftoggle{INCLUDEANSWERS}{{\color{violet!70!white}\textbf{Solution:} #1}}{}}


\begin{document}
  
\section*{Midterm 2 Prep}

\subsection*{Production Functions}
\begin{enumerate}
  \item Consider the production function $Q(K, L) = 5K + L$. 
  \begin{enumerate}
    \item What is the marginal product of capital and labor? Interpret these in words.
    
    \item What is marginal rate of technical substitution $MRTS_{K, L}$? 
    
    \item What does this production function tell you about the substitutability of labor and capital?
    
    \item Does this production function exhibit constant, increasing or decreasing returns to scale?
    
    \item Draw 3 isoquant curves for this production function.
    
    \item Holding fixed capital at $K = 10$, what happens to the marginal product of labor as the firm increases $L$?
  \end{enumerate}

  \answer{
    \begin{enumerate}
      \item $MP_K = 5$ and $MP_L = 1$. For each additional unit of labor, one additional unit of output is produced. For each additional unit of capital used, one additional unit of output is produced.
      
      \item $MRTS_{K,L} = \frac{MP_L}{MP_K} = 1/5$. For every unit of labor the producer gives up, they need 1/5 unit of capital to produce the same output as before.
      
      \item Increasing returns to scale. $Q(\phi K, \phi L) = 5 \phi K \phi L = \phi^2 Q(K, L)$.
      
      \item Lines with a slop of $-1/5$.
      
      \item Stays constant at $1$.
    \end{enumerate}
  }

  \item Consider the production function $Q(K, L) = 5KL$. 
  \begin{enumerate}
    \item What is the marginal product of capital and labor? Interpret these in words.
    
    \item What is marginal rate of technical substitution $MRTS_{K, L}$? 
    
    \item Does this production function exhibit constant, increasing or decreasing returns to scale?
    
    \item Draw 3 isoquant curves for this production function.

    \item Holding fixed capital at $K = 10$, what happens to the marginal product of labor as the firm increases $L$?
    
    \item If the production function becomes $Q(K,L) = 5K^{3/2}L$, is this capital-biased, labor-biased, or neutral technological change?
  \end{enumerate}

  \answer{
    \begin{enumerate}
      \item $MP_K = 5L$ and $MP_L = 5K$. For each additional unit of labor, $5K$ additional unit of output is produced. For each additional unit of capital used, $5L$ additional unit of output is produced.
      
      \item $MRTS_{K,L} = \frac{MP_L}{MP_K} = K/L$. For every unit of labor the producer gives up, they need $K/L$ unit of capital to produce the same output as before.
      
      \item Curvy macaroni shapes
      
      \item Holding $K = 10$, the marignal product of labor increases with $L$ 
      
      \item $MRTS_{K,L} = \frac{MP_L}{MP_K} = \frac{5K^{3/2}}{5 * 3/2 K^{1/2} L} = \frac{2}{3} \frac{K}{L}$. This means, for every unit of labor the producer gives up, they need $\frac{2}{3} K/L$ unit of capital to produce the same output as before. This is less than before, so capital is more productive than before. Therefore this is capital-biased technological change.
    \end{enumerate}
  }

  \item Let $T$ represent car tires and $F$ represent car frames. The production of a car requires 4 tires and 1 frame. 
  \begin{enumerate}
    \item Draw 3 isoquant curves for this production function.
    
    \item Write this as a mathematical production function.
  \end{enumerate}

  \answer{
    \begin{enumerate}
      \item A bunch of "brackets" along the line $4F = T$.
      
      \item $Q(F, T) = \min\{ 4F, T \}$
    \end{enumerate}
  }
\end{enumerate}

\subsection*{Cost Minimization}
\begin{enumerate}
  \item For the following scenarios, find the cost-minimizing input bundles and the total costs of producing the given quantity
  \begin{enumerate}
    \item $Q(K, L) = 2L + 4K$, $r = 8$, $w = 2$, and $\bar{Q} = 40$

    \item $Q(K, L) = 2L + 4K$, $r = 8$, $w = 6$, and $\bar{Q} = 40$

    \item $Q(K, L) = K^{1/2}L^{1/2}$, $r = 8$, $w = 8$, and $\bar{Q} = 40$ 

    \item $Q(K, L) = \min(K, 2L)$, $r = 8$, $w = 2$, and $\bar{Q} = 40$
  \end{enumerate}

  \answer{
    \begin{enumerate}
      \item $\frac{MP_K}{r} = 1/2$ and $\frac{MP_L}{w} = 1$, so use only labor and $K^* = 0$. $2L + 4 * 0 = 40 \implies L^* = 20$. 
      
      \item $\frac{MP_K}{r} = 1/2$ and $\frac{MP_L}{w} = 1/3$, so use only capital and $L^* = 0$. $2*0 + 4K = 40 \implies K^* = 10$. 
      
      \item Our optimality condition is given by
      $$
        \frac{MP_K}{r} = \frac{MP_L}{w} \implies \frac{1/2 K^{-1/2}L^{1/2}}{8} = \frac{1/2 L^{-1/2}K^{1/2}}{8} \implies K = L
      $$

      Plugging this into our isoquant constraint, we have:
      $$
        40 = K^{1/2} K^{1/2} \implies K^* = 40 \text{ and } L^* = 40
      $$

      \item $K^* = 40$ and $2L = 40 \implies L^* = 20$
    \end{enumerate}
  }

  \item Describe in words, why a firm producing with cobb-douglas technology needs to have $MP_k/r = MP_L/w$.

  \answer{The marginal product per dollar of each input has to be equal. If one was larger, then you could save money by using more of the input with a larger marginal product of labor and less of the other input.}
  
  \item A firm is producing the required amount of output, $\bar{Q}$ units with $MP_k / r = 2$ and $MP_L / w = 4$. Is this firm producing at the lowest-possible cost? If not, explain how the firm could shift between inputs and lower costs.
    
  \item Consider the production function $Q(K, L) = KL^{1/2}$.
  \begin{enumerate}
    \item Solve for the conditional input demand functions. 
    
    \item Are labor and capital normal or inferior inputs?
    
    \item In the short-run, the firm's capital is fixed at $\bar{K} = 10$. Find the short-run conditional labor demand function. 
    
    \item What is the labor demanded to produce $\bar{Q} = 20$ units in the short-run when $\bar{K} = 10$.
  \end{enumerate} 

  \answer{
    \begin{enumerate}
      \item The optimality condtion is given by
      $$
        \frac{L^{1/2}}{r} = \frac{1/2L^{-1/2}K}{w} \implies K = 2 \frac{w}{r} L
      $$

      Plugging that into the production constraint gives
      $$
        \bar{Q} = 2 \frac{w}{r} L^{3/2} \implies L^*(\bar{Q}, w, r) = \big(\frac{\bar{Q} r}{2w} \big)^{2/3}
      $$
      Likewise, $K^*(\bar{Q}, w, r) = 2 \frac{w}{r} \big(\frac{\bar{Q} r}{2w} \big)^{2/3}$

      \item They are normal inputs since $\partial L^* / \partial \bar{Q} > 0$ and $\partial K^* / \partial \bar{Q} > 0$.
      
      \item $\bar{Q} = 10 L^{1/2} \implies L^*(\bar{Q}) = \frac{\bar{Q}^2}{100}$
      
      \item $L^*(10) = \frac{10^2}{100} = 1$
    \end{enumerate}
  }
\end{enumerate}

\subsection*{Cost Curves}
\begin{enumerate}
  \item Consider the production function $Q(K, L) = K^{1/2}L^{1/2}$.
  \begin{enumerate}
    \item Let $w = 2$ and $r = 4$. Solve for the total cost function of producing $\bar{Q}$ units.
    
    \item Now, solve for the long-run toral cost curve as a function of $\bar{Q}$, $w$, and $r$.
    
    \item In the short-run, the firm's capital is fixed at $\bar{K} = 16$. Find the short-run conditional labor demand function. 
    
    \item Find the short-run total cost curve when $\bar{K} = 16$. Is it true that $TC(\bar{Q}, w, r) \leq SRTC(\bar{Q}, w, r)$?
    
    \item In words, explain why the short-run total cost curve has to be larger than the long-run total cost curve.
  \end{enumerate} 

  \answer{
    \begin{enumerate}
      \item The optimality condition is $L = 2K$. Plugging this into the production constraint yields
      $$
        \bar{Q} = (2K)^{1/2} K^{1/2} = \sqrt{2} K  \implies K^* = \bar{Q}/\sqrt{2}
      $$
      Pluggin $K^*$ back into the optimality condition yields $L^* = \sqrt{2}\bar{Q}$.

      Thus, the total cost is given by 
      $$
      TC(\bar{Q}) = wL^* + rK^* = 2\sqrt{2} \bar{Q} + 4/\sqrt{2} \bar{Q}
      $$

      \item The optimality condition is $L = \frac{r}{w} K$. Plugging this into the production constraint yields
      $$
        \bar{Q} = (\frac{r}{w} K)^{1/2} K^{1/2} = \sqrt{\frac{r}{w}} K  \implies K^* = \sqrt{\frac{w}{r}} \bar{Q} 
      $$
      Pluggin $K^*$ back into the optimality condition yields $L^* = \sqrt{\frac{w}{r}} \bar{Q}$.

      Thus, the total cost is given by 
      $$
      TC(\bar{Q}) = wL^* + rK^* = w \sqrt{\frac{w}{r}} \bar{Q} + r \sqrt{\frac{w}{r}} \bar{Q}
      $$

      \item The production constraint is given by $\bar{Q} = 16^{1/2} L^{1/2} \implies L^* = \bar{Q}^2/16$
      
      \item The short-run cost curve is given by 
      $$
        SRTC(\bar{Q}, w, r) = r * 16 + w * \bar{Q}^2/16.
      $$

      \item The short-run total cost curve must be weakly larger than the total cost curve since you can not chose the optimal amount of capital in the short-run. Therefore, the optimality condition for cost minimization is not likely to hold.

    \end{enumerate}
  }

  \item Consider the total cost curve $TC(Q) = \frac{Q}{25} \sqrt{wr}$.
  \begin{enumerate}
    \item What is the marginal cost of producing the 11th unit of output when $w = 4, r = 4$?
    
    \item What is the firm's average total cost of producing 40 units when $w = 2$ and $r = 8$? Interpret this in words

    \item Does this firm experience economies of scale?
  \end{enumerate}

  \answer{
    \begin{enumerate}
      \item $MC(Q) = \frac{\sqrt{wr}}{25}$ which when $w = 4$ and $r = 4$ yields $MC(11) = \frac{4}{25} = 0.16$. This means the 11th unit costs 16 cents to make.
      
      \item $ATC(Q) = TC(Q) / Q = \frac{\sqrt{wr}}{25}$. When $w = 2$ and $r = 4$, we have $ATC(40) = \frac{4}{25} = 0.16$. This means the first 40 units cost on average 16 censs to make.
      
      \item Economies of scale is when $\partial ATC(Q) / \partial Q < 0$. In this case, the $ATC$ is constant, so $\partial ATC(Q) / \partial Q = 0$. Therefore, the firm does not experience economies of scale.
    \end{enumerate}
  }

  \item Consider the total cost curve $TC(Q) = Q^2 - 2Q + 10$.
  \begin{enumerate}
    \item What is the average total cost curve, $ATC(Q)$ and the marginal cost curve, $MC(Q)$?
    
    \item What is the quantity where the firm is producing at the minimum efficient scale?
    
    \item When is the firm experiencing economies of scale and diseconomie of scale? (\emph{Hint:} use the previous question) 
    
    \item In the short run, the firm can not change the amount of labor they use because of worker's contracts. What is the relationship between the short-run marginal cost and the long-run marginal cost?
  \end{enumerate}

  \answer{
    \begin{enumerate}
      \item $MC(Q) = 2Q - 2$ and $ATC(Q) = TC(Q) / Q = Q - 2 + 10/Q$. 
      
      \item Minimum efficient scale is when $MC(Q^{MES}) = ATC(Q^{MES})$. This implies
      $$
        2Q - 2 = Q - 2 + 10/Q \implies Q = 10/Q \implies Q^2 = 10 \implies Q^{MES} = \sqrt{10}
      $$

      \item The firm is experiencing economies of scale when $Q < \sqrt{10}$ and diseconomies of scale when $Q > \sqrt{10}$. 
      
      \item In the short run, the firm's short-run marginal cost must be weakly greated than the long-run marginal cost.
    \end{enumerate}
  }
\end{enumerate}


\end{document}
