\documentclass[11pt]{article}
\title{Midterm 2 Prep}
%% Language and font encodings
\usepackage[english]{babel}
\usepackage[utf8x]{inputenc}
\usepackage[T1]{fontenc}

\usepackage{helvet}

%% Sets page size and margins
\usepackage[letterpaper,top=3cm,bottom=2cm,left=3cm,right=3cm,marginparwidth=1.75cm]{geometry}

%% Useful packages
\usepackage{amsmath}
\usepackage{graphicx}
\usepackage{tcolorbox}
\usepackage{amssymb}
\usepackage{amsthm}
\usepackage{lastpage}
\usepackage{accents}
\usepackage{multicol}

% For better list numbering
\usepackage[shortlabels]{enumitem}

% Font
% \usepackage{tgbonum}


% Tikz
\usepackage{tikz}

\usetikzlibrary{calc,fit,shapes.misc,backgrounds}
\usepackage{pgfplots}
\pgfplotsset{compat = newest}
\usetikzlibrary{positioning, arrows.meta}
\usepgfplotslibrary{fillbetween}

% Headers
\usepackage{fancyhdr}
\pagestyle{fancy}

% Store \@title as \thetitle
\makeatletter
\let\thetitle\@title
\makeatother

\fancyhf{}
\lhead{\fontfamily{qbk}\fontsize{10}{11}\selectfont ECON 3070}
\rhead{\fontfamily{qbk}\fontsize{10}{11}\selectfont \thetitle}
\rfoot{\fontfamily{qbk}\fontsize{10}{11}\selectfont \thepage}


% Sections and Subsections

% define colors
\definecolor{buff-gold}{HTML}{CFB87C}
\definecolor{buff-grey}{HTML}{565A5C}
% custom tcolorbox
\tcbset{colframe=buff-gold, colback=white!100!black}

% new page per section
\usepackage{titlesec}
\newcommand{\sectionbreak}{\clearpage}
% change style of section
\usepackage{sectsty}
\sectionfont{\color{buff-gold} \fontfamily{qbk}\selectfont}
\subsectionfont{\color{buff-grey} \fontfamily{qbk}\selectfont}


\newtoggle{INCLUDEANSWERS}
% \toggletrue{INCLUDEANSWERS}
\togglefalse{INCLUDEANSWERS}

\newcommand{\answer}[1]{\iftoggle{INCLUDEANSWERS}{{\color{violet!70!white}\textbf{Solution:} #1}}{}}


\begin{document}
  
\section*{Midterm 2 Prep}

\subsection*{Production Functions}
\begin{enumerate}
  \item Consider the production function $Q(K, L) = 5K + L$. 
  \begin{enumerate}
    \item What is the marginal product of capital and labor? Interpret these in words.
    
    \item What is marginal rate of technical substitution $MRTS_{K, L}$? 
    
    \item What does this production function tell you about the substitutability of labor and capital?
    
    \item Does this production function exhibit constant, increasing or decreasing returns to scale?
    
    \item Draw 3 isoquant curves for this production function.
    
    \item Holding fixed capital at $K = 10$, what happens to the marginal product of labor as the firm increases $L$?
  \end{enumerate}

  \item Consider the production function $Q(K, L) = 5KL$. 
  \begin{enumerate}
    \item What is the marginal product of capital and labor? Interpret these in words.
    
    \item What is marginal rate of technical substitution $MRTS_{K, L}$? 
    
    \item Does this production function exhibit constant, increasing or decreasing returns to scale?
    
    \item Draw 3 isoquant curves for this production function.

    \item Holding fixed capital at $K = 10$, what happens to the marginal product of labor as the firm increases $L$?
    
    \item If the production function becomes $Q(K,L) = 5K^{3/2}L$, is this capital-biased, labor-biased, or neutral technological change?
  \end{enumerate}

  \item Let $T$ represent car tires and $F$ represent car frames. The production of a car requires 4 tires and 1 frame. 
  \begin{enumerate}
    \item Draw 3 isoquant curves for this production function.
    
    \item Write this as a mathematical production function.
  \end{enumerate}
\end{enumerate}

\subsection*{Cost Minimization}
\begin{enumerate}
  \item For the following scenarios, find the cost-minimizing input bundles and the total costs of producing the given quantity
  \begin{enumerate}
    \item $Q(K, L) = 2L + 4K$, $r = 8$, $w = 2$, and $\bar{Q} = 40$

    \item $Q(K, L) = 2L + 4K$, $r = 8$, $w = 6$, and $\bar{Q} = 40$

    \item $Q(K, L) = K^{1/2}L^{1/2}$, $r = 8$, $w = 8$, and $\bar{Q} = 40$ 

    \item $Q(K, L) = \min(K, 2L)$, $r = 8$, $w = 2$, and $\bar{Q} = 40$
  \end{enumerate}

  \item Describe in words, why a firm producing with cobb-douglas technology needs to have $MP_k/r = MP_L/w$.
  
  \item A firm is producing the required amount of output, $\bar{Q}$ units with $MP_k / r = 2$ and $MP_L / w = 4$. Is this firm producing at the lowest-possible cost? If not, explain how the firm could shift between inputs and lower costs.
    
  \item Consider the production function $Q(K, L) = KL^{1/2}$.
  \begin{enumerate}
    \item Solve for the conditional input demand functions. 
    
    \item Are labor and capital normal or inferior inputs?
    
    \item In the short-run, the firm's capital is fixed at $\bar{K} = 10$. Find the short-run conditional labor demand function. 
    
    \item What is the labor demanded to produce $\bar{Q} = 20$ units in the short-run when $\bar{K} = 10$.
  \end{enumerate} 
\end{enumerate}

\subsection*{Cost Curves}
\begin{enumerate}
  \item Consider the production function $Q(K, L) = K^{1/2}L^{1/2}$.
  \begin{enumerate}
    \item Let $w = 2$ and $r = 4$. Solve for the total cost function of producing $\bar{Q}$ units.
    
    \item Now, solve for the long-run toral cost curve as a function of $\bar{Q}$, $w$, and $r$.
    
    \item In the short-run, the firm's capital is fixed at $\bar{K} = 16$. Find the short-run conditional labor demand function. 
    
    \item Find the short-run total cost curve when $\bar{K} = 16$. Is it true that $TC(\bar{Q}, w, r) \leq SRTC(\bar{Q}, w, r)$?
    
    \item In words, explain why the short-run total cost curve has to be larger than the long-run total cost curve.
  \end{enumerate} 

  \item Consider the total cost curve $TC(Q) = \frac{Q}{25} \sqrt{wr}$.
  \begin{enumerate}
    \item What is the marginal cost of producing the 11th unit of output when $w = 4, r = 4$?
    
    \item What is the firm's average total cost of producing 40 units when $w = 2$ and $r = 8$? Interpret this in words

    \item Does this firm experience economies of scale?
  \end{enumerate}

  \item Consider the total cost curve $TC(Q) = Q^2 - 2Q + 10$.
  \begin{enumerate}
    \item What is the average total cost curve, $ATC(Q)$ and the marginal cost curve, $MC(Q)$?
    
    \item What is the quantity where the firm is producing at the minimum efficient scale?
    
    \item When is the firm experiencing economies of scale and diseconomie of scale? (\emph{Hint:} use the previous question) 
    
    \item In the short run, the firm can not change the amount of labor they use because of worker's contracts. What is the relationship between the short-run marginal cost and the long-run marginal cost?
  \end{enumerate}
\end{enumerate}


\end{document}
